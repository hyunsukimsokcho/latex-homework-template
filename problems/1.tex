\begin{homeworkProblem}
    (a) Using Integration by parts,
    \begin{center}
        \(\E[X] = \int_{-\infty}^{\infty} xf(x) \dx = \int_{-\infty}^{\theta} x \cdot 0 \dx + \int_{\theta}^{\infty} xe^{\theta - x} \dx = 1 + \theta \) \\
        \vspace{3mm}
        \(\Var[X] = \E[X^2] - \E[X]^2 = \int_{\theta}^{\infty} x^2e^{\theta - x} \dx - (1 + \theta)^2 = \theta^2 + 2(1 + \theta) - (1 + \theta)^2 = 1 \)
    \end{center}
\vspace{3mm}
    (b) To show unbiasedness, compare $\theta$ with the exepectation. Note $X_i$'s are i.i.d., and $\E[X_i] = 1 + \theta$ in (a):
    \[
        \begin{split}
            \E[\hat{\theta}_n] &= \E[\frac{1}{n}\sum^{n}_{i=1}(X_i - 1)]
            \\
            &= \frac{1}{n}\sum^{n}_{i=1}\E[X_i - 1]
            \\
            &= \frac{1}{n}\sum^{n}_{i=1}(\E[X_i] - 1)
            \\
            &= \frac{1}{n}\sum^{n}_{i=1}((1 + \theta) - 1)
            \\
            &= \frac{1}{n}\sum^{n}_{i=1}\theta
            \\
            &= \frac{1}{n}(n\theta)
            \\
            &= \theta
        \end{split}
    \]
\vspace{3mm}
    (c) Let $Y_i = X_i - 1$. Then, $\E[Y_i] = \theta$ and $\Var[Y_i] = 1$. Now, according to (b) and Central Limit Theorem,
    \[
        \frac{\hat{\theta}_n - \theta}{1/\sqrt{n}} \xrightarrow{d} \mathcal{N}(0, 1)
    \]
    \\
    holds. Then, $100(1-\alpha) \% $ confidence interval of $\theta$ is 
    $[\hat{\theta}_n - \frac{\Phi^{-1}(1-\frac{\alpha}{2})}{\sqrt{n}}, \hat{\theta}_n - \frac{\Phi^{-1}(\frac{\alpha}{2})}{\sqrt{n}}]$
    where $\Phi$ is Cumulative Density Function (CDF) of unit normal distribution.
    \\\\
\vspace{1mm}
    (d) From observations, we can easily compute $\hat{\theta}_3 = \frac{1}{3}\{(10.0 - 1) + (12.0 - 1) + (15.0 - 1)\} = 11.33$\,. 
    According to given fact and that unit normal distribution is an even function, 
    $\Phi^{-1}(0.025) = -1.96$ and $\Phi^{-1}(0.975) = 1.96$. Thus, 95\% confidence interval of 
    $\theta$ is $[11.33 - \frac{1.96}{\sqrt{3}}, 11.33 + \frac{1.96}{\sqrt{3}}] = [10.20, 12.46]$.
    It is weird that the observed data 10.0 is contradictory for any $\theta$ in the obtained confidence interval according to the PDF of $X$.
    Such odd situation is indeed expected to happen because of the small sample size. ($n = 3$) 
    Also, the computed confidence interval is based on frequentist approach, of which ``95\%'' stands for how frequently
    would $\theta$ be contained in the interval as we repeat the procedure of computing confidence interval.
\end{homeworkProblem}